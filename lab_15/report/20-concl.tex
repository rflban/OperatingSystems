\chapter{Структура FILE}

В листинге~\ref{lst:file} приведено определение структуры FILE.

\begin{lstlisting}[caption={Структура FILE},label=lst:file]
// /usr/include/x86_64-linux-gnu/bits/types/FILE.h
#ifndef __FILE_defined
#define __FILE_defined 1

struct _IO_FILE;

/* The opaque type of streams.  This is the definition used elsewhere.  */
typedef struct _IO_FILE FILE;

#endif

// /usr/include/x86_64-linux-gnu/bits/types/struct_FILE.h
/* Copyright (C) 1991-2018 Free Software Foundation, Inc.
   This file is part of the GNU C Library.

   The GNU C Library is free software; you can redistribute it and/or
   modify it under the terms of the GNU Lesser General Public
   License as published by the Free Software Foundation; either
   version 2.1 of the License, or (at your option) any later version.

   The GNU C Library is distributed in the hope that it will be useful,
   but WITHOUT ANY WARRANTY; without even the implied warranty of
   MERCHANTABILITY or FITNESS FOR A PARTICULAR PURPOSE.  See the GNU
   Lesser General Public License for more details.

   You should have received a copy of the GNU Lesser General Public
   License along with the GNU C Library; if not, see
   <http://www.gnu.org/licenses/>.  */

#ifndef __struct_FILE_defined
#define __struct_FILE_defined 1
// ...
/* The tag name of this struct is _IO_FILE to preserve historic
   C++ mangled names for functions taking FILE* arguments.
   That name should not be used in new code.  */
struct _IO_FILE
{
  int _flags;		/* High-order word is _IO_MAGIC; rest is flags. */

  /* The following pointers correspond to the C++ streambuf protocol. */
  char *_IO_read_ptr;	/* Current read pointer */
  char *_IO_read_end;	/* End of get area. */
  char *_IO_read_base;	/* Start of putback+get area. */
  char *_IO_write_base;	/* Start of put area. */
  char *_IO_write_ptr;	/* Current put pointer. */
  char *_IO_write_end;	/* End of put area. */
  char *_IO_buf_base;	/* Start of reserve area. */
  char *_IO_buf_end;	/* End of reserve area. */

  /* The following fields are used to support backing up and undo. */
  char *_IO_save_base; /* Pointer to start of non-current get area. */
  char *_IO_backup_base;
  /* Pointer to first valid character of backup area */
  char *_IO_save_end; /* Pointer to end of non-current get area. */

  struct _IO_marker *_markers;

  struct _IO_FILE *_chain;

  int _fileno;
  int _flags2;
  __off_t _old_offset; /* This used to be _offset but it's too small.  */

  /* 1+column number of pbase(); 0 is unknown. */
  unsigned short _cur_column;
  signed char _vtable_offset;
  char _shortbuf[1];

  _IO_lock_t *_lock;
#ifdef _IO_USE_OLD_IO_FILE
};

struct _IO_FILE_complete
{
  struct _IO_FILE _file;
#endif
  __off64_t _offset;
  /* Wide character stream stuff.  */
  struct _IO_codecvt *_codecvt;
  struct _IO_wide_data *_wide_data;
  struct _IO_FILE *_freeres_list;
  void *_freeres_buf;
  size_t __pad5;
  int _mode;
  /* Make sure we don't get into trouble again.  */
  char _unused2[15 * sizeof (int) - 4 * sizeof (void *) - sizeof (size_t)];
};
// ...
#endif
\end{lstlisting}

